\documentclass[a4paper,10pt]{book}
\usepackage[utf8x]{inputenc}

\begin{document}

\chapter{develop algorithms}

\section{develop Graphical Algorithms}

\paragraph*{}
Graphical algorithms open graphical elements during their execution. They intend to display something as a window integrated within genlab.

\paragraph*{}
These graphical algorithms are made of two parts:
\begin{itemize}
\item the view is the eclispe RCP view part, that is the actual window integrated inside the GUI,
\item the algorithm is the genlab algorithm which is going to open a windows
\end{itemize}

\subsection{View}

\paragraph*{}
A view should:
\begin{itemize}
 \item inherit the class ``AbstractViewOpenedByAlgo''.
\item always have an unique ID defined as a static field ID of its class
\item be declared against eclipse using the extension org.eclipse.ui.views
\item be associated with a position in the genlab perspectives using the extension ``org.eclipse.ui.perspectiveExstension''
\item 
\end{itemize}


\section{propose example}

To provide an example to a user:
\begin{itemize}
\item in your algorithm plugin (i.e. the corresponding eclipse project), create a YOURPLUGINPACKAGES.examples (just a convention to make things more clear)
\item create a different class for every example which implements IGenlabExample
\item in the plugin.xml, section ``extensions'', add an extension point genlab.examples.extensions.example; then inside it register each class providing an example
\end{itemize}

 
 

\end{document}
